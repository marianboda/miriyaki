\chapter*{Úvod}
\addcontentsline{toc}{chapter}{Úvod}

Zvuk syntetizátorov je bežnou súčasťou mnohých hudobných štýlov už viac ako tridsať rokov. Počas tejto doby sa ich zvuk, ale aj forma ich existencie výrazne menili. V~dnešnej dobe, keď sú počítače už takmer nevyhnutnou súčasťou každej domácnosti, je použitie syntetizátorov prístupnejšie oveľa širšiemu okruhu hudobných nadšencov. Rovnako dostupná je aj možnosť vytvorenia vlastného virtuálneho nástroja, a~to už nielen pre pokročilých programátorov, ale aj pre bežných používateľov, ktorí majú základné vedomosti o zvukovej syntéze. Pre nich existuje niekoľko vizuálnych prostredí, ktoré im poskytujú dostatok nástrojov na realizáciu svojich predstáv. Pre tých náročnejších však stále zostáva len cesta kompletného programovania.

V tejto diplomovej práci sa snažím zachytiť základné princípy fungovania syntetizátorov a ich stručnú históriu. Usilujem sa vysvetliť základné princípy fungovania rôznych typov syntézy, zameriavam sa na softvérovú syntézu a predstavujem najpoužívanejšie rozhrania, pre ktoré sa syntetizátory vyvíjajú. Ďalej opisujem rozhranie VST a spôsoby tvorby syntetizátorov pre toto rozhranie. V ďalších kapitolách bližšie rozoberám aktuálny stav voľne šíriteľných subtraktívnych syntetizátorov pre rozhranie VST a nakoniec ukazujem na príklade návrh a implementáciu subtraktívneho VST syntetizátora v jazyku C++. Pri návrhu tohto nástroja kladiem dôraz hlavne na kvalitu implementovaných častí a na príjemnosť a praktickosť používateľského rozhrania. Podrobnejšie sa venujem problematike generovania antialiasovaných signálov, návrhu digitálneho filtra a optimalizácii rýchlosti kódu. Na konci práce vyhodnocujem výsledok implementácie spolu s odbornými názormi hudobníkov.
